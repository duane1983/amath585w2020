
\documentclass[11pt]{article}


\usepackage{amsmath,amsfonts,amssymb}

\setlength{\textwidth}{7.2in}
\setlength{\oddsidemargin}{-0.5in}
\setlength{\evensidemargin}{-0.5in}
\setlength{\textheight}{9.3in}
\setlength{\voffset}{-.9in}
\setlength{\headsep}{26pt}

%\input{/Users/rjl/Dropbox/tex/RJLmacros}

% a few handy macros

\newcommand\matlab{{\sc matlab}}
\newcommand{\goto}{\rightarrow}
\newcommand{\bigo}{{\mathcal O}}
\newcommand{\half}{\frac{1}{2}}
%\newcommand\implies{\quad\Longrightarrow\quad}
\newcommand\reals{{{\rm l} \kern -.15em {\rm R} }}
\newcommand\complex{{\raisebox{.043ex}{\rule{0.07em}{1.56ex}} \hskip -.35em {\rm C}}}


% macros for matrices/vectors:

% matrix environment for vectors or matrices where elements are centered
\newenvironment{mat}{\left[\begin{array}{ccccccccccccccc}}{\end{array}\right]}
\newcommand\bcm{\begin{mat}}
\newcommand\ecm{\end{mat}}

% matrix environment for vectors or matrices where elements are right justifvied
\newenvironment{rmat}{\left[\begin{array}{rrrrrrrrrrrrr}}{\end{array}\right]}
\newcommand\brm{\begin{rmat}}
\newcommand\erm{\end{rmat}}

% for left brace and a set of choices
\newenvironment{choices}{\left\{ \begin{array}{ll}}{\end{array}\right.}
\newcommand\when{&\text{if~}}
\newcommand\otherwise{&\text{otherwise}}
% sample usage:
%  \delta_{ij} = \begin{choices} 1 \when i=j, \\ 0 \otherwise \end{choices}


% for labeling and referencing equations:
\newcommand{\eql}{\begin{equation}\label}
\newcommand{\eqn}[1]{(\ref{#1})}
% can then do
%  \eql{eqnlabel}
%  ...
%  \end{equation}
% and refer to it as equation \eqn{eqnlabel}.  


% some useful macros for finite difference methods:
\newcommand\unp{U^{n+1}}
\newcommand\unm{U^{n-1}}

% for chemical reactions:
\newcommand{\react}[1]{\stackrel{K_{#1}}{\rightarrow}}
\newcommand{\reactb}[2]{\stackrel{K_{#1}}{~\stackrel{\rightleftharpoons}
   {\scriptstyle K_{#2}}}~}

% Parts:

% set enumerate to give parts a, b, c, ...  rather than numbers 1, 2, 3...
\renewcommand{\theenumi}{\alph{enumi}}
\renewcommand{\labelenumi}{(\theenumi)}

% set second level enumerate to give parts i, ii, iii, iv, etc.
\renewcommand{\theenumii}{\roman{enumii}}
\renewcommand{\labelenumii}{(\theenumii)}



\begin{document}


\hfill\vbox{\hbox{\bf AMath 585}
\hbox{Winter Quarter, 2020}
\hbox{R.\ J.\ LeVeque}}


\vskip 30pt
\centerline{\Large\bf Fourier Transforms}
\vskip 10pt



This is a summary of some key facts about Fourier integrals, series, 
sums, and transforms,
and the manner in which these concepts relate to one another.
Many different notations are used in the literature and in software for
Fourier transforms, so it is important to make sure you understand the
notation and scaling used in the paper/book/software you are using.


\vskip 5pt
In the ``physical domain''
I will use $u(x)$ to denote a function of $x$ defined for some interval
of $x$ values (possibly all real $x$) and $u_j$ to denote a discrete set of
values over some set (finite or infinite) of integer indices~$j$.

In the ``Fourier domain'' 
I will use $\hat u(\xi)$ to denote a function of $\xi$ defined for some interval
of wave number values (possibly all real $\xi$) 
and $\hat u_k$ to denote a discrete set of wave number 
values over some set (finite or infinite) of integer indices $k$.

\vskip 5pt
There are four cases to consider:
\newcounter{case}
\begin{list}{{\bf Case \arabic{case}.}}{\usecounter{case}
             \setlength{\leftmargin}{1.5cm}}
\item The data is an $L_2$ function $u(x)$ defined for all real $x$.
\item The data is a function $u(x)$ defined on a finite interval $0
< x < 2\pi$ or $-\pi < x < \pi$ (or a periodic function defined on one period).
\item The data is a discrete grid function $u_j$ on an infinite grid, at
points $x_j = jh$ for $j=0,~\pm 1,~\pm2,~\ldots$ and some $h>0$.
\item The data is a discrete grid function $u_j$ on a grid with $N= 2\pi / h$
points in the interval $[-\pi,\pi]$ (or a periodic grid function
defined on one period).
\end{list}

The Fourier transform $\hat u$ has a different form in each case, as listed
below for each of case.  Note the duality: the data in Case 2 has the same
form as the transform in Case 3 and vice versa.  Cases 1 and 4 are
self-dual; the data and transform have the same general form.

\begin{list}{{\bf Case \arabic{case}.}}{\usecounter{case}
             \setlength{\leftmargin}{1.5cm}}
\item The Fourier transform $\hat u(\xi)$ is an $L_2$
function defined for all real $\xi$.
\item The Fourier transform $\hat u_k$ is a set of discrete values
defined on an infinite grid, for wave numbers $\xi_k =
k=0,~\pm 1,~\pm2,~\ldots$.
\item The Fourier transform $\hat u(\xi)$ is a function defined on a finite
interval $[-\pi/h,~\pi/h]$.
\item The Fourier transform $\hat u_k$ is a set of $N$ discrete values for
wave numbers \\
$\xi_k = k = -N/2 + 1, ~\ldots,~N/2$.
This grid covers the interval $[-\pi/h,~\pi/h]$.
\end{list}


The formulas in each case are given on the next page.  One particular
normalization has been chosen, more about this below.


\newpage
\centerline{\large\bf Fourier transform and inverse transform formulas:}
\vskip 5pt

\noindent
{\bf Case 1.} The data is an $L_2$ function $u(x)$ defined for all real $x$.
The Fourier transform $\hat u(\xi)$ is an $L_2$
function defined for all real $\xi$.

\vskip 5pt{\bf Forward Transform:}
\eql{1a}\tag{1a}
\hat u(\xi) = \int_{-\infty}^{\infty} e^{-i\xi x}\,u(x)\,dx \qquad\text{for~}
-\infty<\xi<\infty.
\end{equation} 

{\bf Inverse transform:}
\eql{1b}\tag{1b}
u(x) = \frac{1}{2\pi} \int_{-\infty}^{\infty} e^{i\xi x}\,\hat u(\xi)\,d\xi
\qquad\text{for~} -\infty < x < \infty.
\end{equation} 

\noindent
{\bf Case 2.} The data is a function $u(x)$ defined on a finite interval $0
< x < 2\pi$ (or a periodic function defined on one period).
The Fourier transform $\hat u_k$ is a set of discrete values
defined on an infinite grid, for wave numbers $\xi_k =
k=0,~\pm 1,~\pm2,~\ldots$.
This grid covers the interval $[-\pi/h,~\pi/h]$.

\vskip 5pt{\bf Forward Transform:}
\eql{2a}\tag{2a}
\hat u_k = \int_0^{2\pi} e^{-ik x}\,u(x)\,dx \qquad\text{for~}k=0,~\pm 1,~\pm
2,~\ldots.
\end{equation}

{\bf Inverse transform:}
\eql{2b}\tag{2b}
u(x) = \frac{1}{2\pi} \sum_{k=-\infty}^{\infty} e^{ik x}\,\hat u_k
\qquad\text{for~} 0< x < 2\pi.
\end{equation}


\noindent
{\bf Case 3.} The data is a discrete grid function $u_j$ on an infinite grid, at
points $x_j = jh$ for $j=0,~\pm 1,~\pm2,~\ldots$ and some $h>0$.
The Fourier transform $\hat u(\xi)$ is a function defined on a finite
interval $[-\pi/h,~\pi/h]$.

\vskip 5pt{\bf Forward Transform:}
\eql{3a}\tag{3a}
\hat u(\xi) = \sum_{j=-\infty}^{\infty} e^{-i\xi x_j}\,u_j \qquad\text{for~}
-\frac{\pi}{h} < \xi < \frac{\pi}{h}.
\end{equation}

{\bf Inverse transform:}
\eql{3b}\tag{3b}
u_j = \frac{1}{2\pi} \int_{-\pi/h}^{\pi/h} e^{i\xi x_j}\,\hat u(\xi)\,d\xi
\qquad\text{for~} j=0,~\pm 1,~\pm2,~\ldots.
\end{equation}

\noindent
{\bf Case 4.} The data is a discrete grid function $u_j$ on a grid with $N= 2\pi / h$
points in the interval $[0,2\pi]$ (or a periodic grid function
defined on one period).  The grid points are $x_j = jh$ for
$j=1,~2,~\ldots,~N$, where $h = 2\pi / N$.
The Fourier transform $\hat u_k$ is a set of $N$ discrete values for
wave numbers $\xi_k = k = -N/2 + 1, ~\ldots,~N/2$.
The formulas are written here for the case where $N$ is even.

\vskip 5pt{\bf Forward Transform:}
\eql{4a}\tag{4a}
\hat u_k = h\sum_{j=1}^{N} e^{-ik x_j}\,u_j \qquad\text{for~} 
k=-\frac N 2 + 1,~ \ldots,~\frac N 2. 
\end{equation}

{\bf Inverse transform:}
\eql{4b}\tag{4b}
u_j = \frac{1}{2\pi} \sum_{k=-\frac N 2 +1}^{\frac N 2 } e^{ik x_j}\,\hat u_k
\qquad\text{for~} j=1,~2,~\ldots,~N.
\end{equation}


\newpage
\noindent 
Some things to note and help motivate the different forms:
\begin{itemize}
\item If the data is a function $u(x)$, then this data may contain
arbitrarily high wave number oscillations, and so the transform $\hat u$ may
be nonzero for arbitrarily large values of
$|\xi|$ (in Case 1) or $|k|$ (in Case 2).

\item If the data consists of discrete values then there is a limit to how
high a wave number can be represented on the grid, and so in Cases 3 and 4
the Fourier transform is ``band limited'' and the wave number does not
exceed $\pi / h$ in magnitude.  
%The inverse transform formulas (3b) or (4b)
%could be evaluated at values of $x$ other than the $x_j$, resulting in
%{\it band-limited interpolants} of the original grid data.

\item If the data $u(x)$ or $u_j$ is given only on a finite interval (which
is viewed as
one period of periodic data), as in Cases 2 and 4, then the Fourier
transform $\hat u_k$ takes discrete values since $e^{i\xi x}$ satisfies
the periodicity requirement only for integer wave numbers $\xi_k=k$. 
%The inverse transform formulas (2b) or (4b) could be evaluated for values of
%$x$ or $j$ outside the original interval where the data was given, and the
%resulting function is the {\it periodic extension} of the original data.

\item If the data $u(x)$ or $u_j$ is given over an infinite interval
(non-periodic), as in Cases 1 and 3,
then there is no periodicity requirement constraining $\xi$ to integers
and the Fourier transform $\hat u(\xi)$ is defined over
an interval of $\xi$ values. 
\end{itemize}

\vskip 10pt
\noindent 
{\bf Normalization.}
\vskip 5pt
In each case you can modify the definition of $\hat u$ by
multiplying by a constant $C$ and then the formula for $u(x)$ must be
multiplied by $1/C$ to cancel this factor out.  In particular, if the
formulas above for the Fourier transform
$\hat u$ are multiplied by $\frac 1 {2\pi}$, then the
inverse transform formulas for $u$ will be multiplied by $2\pi$, with the
result that the factor $\frac 1{2\pi}$ will appear in the formulas for 
for $u$ rather than $\hat u$.  This alternative is often seen in the
literature.

Another possibility is to use the factor $C = 1/\sqrt{2\pi}$, with the
result that both the formulas for $u$ and the formulas for $\hat u$ 
will contain a
factor $1/\sqrt{2\pi}$. This is sometimes used, particularly in Case 1,
since it gives a more symmetric and self-dual formula.  However, it also
leads to constantly writing $1/\sqrt{2\pi}$ in every formula.

\vskip 10pt
\noindent{\bf Discrete Fourier transforms.}
\vskip 5pt
Case 4 is often used computationally:  a finite set of $N$ data values $u_j$
is transformed into a finite set of $N$ values $\hat u_k$.  
This is often called the {\em discrete Fourier transform} (DFT).  
The formulas for the DFT and inverse DFT (IDFT) each involve finite sums and
can be evaluated directly (unlike the other cases, which involve integrals
that one may not be able to evaluate exactly).

The formula for each $\hat u_k$ is a sum of $N$ terms, so 
computing this value for one $k$ requires $\bigo(N)$ floating point operations.
Hence it appears that computing all $N$ components of the vector $\hat u$ 
will require $N$ times as much work, or $\bigo(N^2)$ operations.
Note that the process of going from a vector $u$ of $N$ data values to the
vector $\hat u$ of $N$ transform values can be interpreted as simply
multiplying by an $N\times N$
matrix $R$ with $(k,j)$ element equal to $he^{-ikjh}$.
Since this is a dense matrix, this product computed in the standard way
requires $\bigo(N^2)$ operations.

However, this matrix has a very special structure that can be exploited to
compute this particular matrix-vector product in far fewer operations.
This works best if $N$ is a power of 2, or more generally of small primes.
The fast algorithm can be derived in various ways.  The main idea is that
if $N=2^s$ then the matrix $R$ can be written as the product of $s = \log_2
N$ other matrices, each of which is very sparse: $R = R_sR_{s-1}\cdots R_1$.
We can compute the product $Ru$ by first multiplying $u$ by $R_1$, then 
multiplying the resulting vector by
$R_2$, etc.  Each of these matrix-vector products only requires $\bigo(N)$
operations, because of the sparsity, and so computing $Ru$ in this way only
requires $\bigo(N\log N)$ operations.  This algorithm is called the {\em
Fast Fourier Transform} (FFT).

For some applications, very large values of $N$ are used and use of the FFT
results in huge savings (by nearly a factor $N$ over the naive algorithm).
For spectral methods this is usually not such an issue since relatively
small values of $N$ are often used, but it still makes sense to use FFT
software rather than using the slow algorithm.  Not only is it faster, it is
also often more stable numerically.

For the DFT (including FFT software) somewhat different forms of Case 4 above
are often seen.  In particular, redefining $\hat u_k$ by multiplying the 
formula \eqn{4a} by $1/h$ and the formula \eqn{4b} for $u_j$ by $h$ gives the
alternative formulation

\vskip 5pt
\noindent{\bf Case 4 (Alternative scaling).}
\vskip 5pt{\bf Forward Transform:}
\eql{5a}\tag{5a}
\hat u_k = \sum_{j=1}^{N} e^{-ik x_j}\,u_j \qquad\text{for~} 
k=-\frac N 2 + 1,~ \ldots,~\frac N 2. 
\end{equation}

{\bf Inverse transform:}
\eql{5b}\tag{5b}
u_j = \frac 1 N \sum_{k=-\frac N 2 +1}^{\frac N 2 } e^{ik x_j}\,\hat u_k
\qquad\text{for~} j=1,~2,~\ldots,~N.
\end{equation}
where we have used $h/2\pi = 1/N$ to simplify the factor in the transform
\eqn{5a}.

\vskip 5pt
\noindent
Another form often used is a variant of this:

\vskip 5pt
\noindent{\bf Case 4 (Alternative indexing of wave numbers).}
\vskip 5pt{\bf Forward Transform:}
\eql{6a}\tag{6a}
\hat u_k = \sum_{j=0}^{N-1} e^{-ik x_j}\,u_j 
\qquad\text{for~} k=0,~1,~\ldots,~N-1.
\end{equation}

{\bf Inverse transform:}
\eql{6b}\tag{6b}
u_j = \frac 1 N \sum_{k=0}^{N-1} e^{ik x_j}\,\hat u_k.
\qquad\text{for~} j=0,~1,~\ldots,~N-1.
\end{equation}
This form has the virtue of maximum simplicity, and is used in the
Matlab {\tt fft} or Python {\tt numpy.fft} routines, for example.  
Recall that in these formulas
\[
x_j = jh = \frac{2\pi j}{N}.
\]  
Note that essentially
the same formula as in \eqn{5a} is used for the transform, but for a
different set of $k$ values.  By the assumed periodicity,  $u_0 =u_N$
and the change in the limits of summation
in \eqn{6a} is made just to be consistent with
\eqn{6b}.  The change of indices from \eqn{5b} to \eqn{6b} appears more
substantial, but in fact 
$\hat u_{-k} = \hat u_{N-k}$
so this is just a relabelling.  This relation follows from
\begin{equation*}
\begin{split}
\hat u_{-k} &= \sum_j \exp(-i(-k)2\pi j/N) u_j\\
&= \sum_j \exp(-i(N-k)2\pi j/N) \exp(iN2\pi j / N) u_j\\
&= \sum_j \exp(-i(N-k)2\pi j/N)  u_j\\
& = \hat u_{N-k}
\end{split}
\end{equation*}
since $\exp(iN2\pi j / N) = (e^{2\pi i})^j = 1$. So the following two
vectors of transforms are equivalent:
\begin{equation*} 
\begin{split}
&\hat u_0,~~\hat u_1,~~ \ldots,~~\hat u_{N/2},~~\hat
u_{N/2+1},~~\ldots,~~\hat u_{N-1}\\
&\hat u_0,~~\hat u_1,~~ \ldots,~~\hat u_{N/2},~~\hat
u_{-N/2+1},~~\ldots,~~\hat u_{-1}.
\end{split}
\end{equation*} 

The FFTW package (often used in Fortran, C, Python, etc.)
computes unnormalized versions of (6), with no factor of
$1/N$ in either term.  So applying the FFT and then IFFT with this 
software gives the original sequence multiplied by a factor $N$.


\end{document}
