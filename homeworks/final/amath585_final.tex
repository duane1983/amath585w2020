\documentclass[10pt]{article}

\usepackage{graphicx}
\usepackage{amsmath,amsfonts,amssymb}

\usepackage{hyperref}  % for urls and hyperlinks


\setlength{\textwidth}{6.2in}
\setlength{\oddsidemargin}{0.3in}
\setlength{\evensidemargin}{0in}
\setlength{\textheight}{9.1in}
\setlength{\voffset}{-1in}
\setlength{\headsep}{26pt}
\setlength{\parindent}{0pt}
\setlength{\parskip}{5pt}




% a few handy macros

\newcommand\matlab{{\sc matlab}}
\newcommand{\goto}{\rightarrow}
\newcommand{\bigo}{{\mathcal O}}
\newcommand{\half}{\frac{1}{2}}
%\newcommand\implies{\quad\Longrightarrow\quad}
\newcommand\reals{{{\rm l} \kern -.15em {\rm R} }}
\newcommand\complex{{\raisebox{.043ex}{\rule{0.07em}{1.56ex}} \hskip -.35em {\rm C}}}


% macros for matrices/vectors:

% matrix environment for vectors or matrices where elements are centered
\newenvironment{mat}{\left[\begin{array}{ccccccccccccccc}}{\end{array}\right]}
\newcommand\bcm{\begin{mat}}
\newcommand\ecm{\end{mat}}

% matrix environment for vectors or matrices where elements are right justifvied
\newenvironment{rmat}{\left[\begin{array}{rrrrrrrrrrrrr}}{\end{array}\right]}
\newcommand\brm{\begin{rmat}}
\newcommand\erm{\end{rmat}}

% for left brace and a set of choices
\newenvironment{choices}{\left\{ \begin{array}{ll}}{\end{array}\right.}
\newcommand\when{&\text{if~}}
\newcommand\otherwise{&\text{otherwise}}
% sample usage:
%  \delta_{ij} = \begin{choices} 1 \when i=j, \\ 0 \otherwise \end{choices}


% for labeling and referencing equations:
\newcommand{\eql}{\begin{equation}\label}
\newcommand{\eqn}[1]{(\ref{#1})}
% can then do
%  \eql{eqnlabel}
%  ...
%  \end{equation}
% and refer to it as equation \eqn{eqnlabel}.  


% some useful macros for finite difference methods:
\newcommand\unp{U^{n+1}}
\newcommand\unm{U^{n-1}}

% for chemical reactions:
\newcommand{\react}[1]{\stackrel{K_{#1}}{\rightarrow}}
\newcommand{\reactb}[2]{\stackrel{K_{#1}}{~\stackrel{\rightleftharpoons}
   {\scriptstyle K_{#2}}}~}

% Parts:

% set enumerate to give parts a, b, c, ...  rather than numbers 1, 2, 3...
\renewcommand{\theenumi}{\alph{enumi}}
\renewcommand{\labelenumi}{(\theenumi)}

% set second level enumerate to give parts i, ii, iii, iv, etc.
\renewcommand{\theenumii}{\roman{enumii}}
\renewcommand{\labelenumii}{(\theenumii)}

  % input some useful macros

\begin{document}

% header:
\hfill\vbox{\hbox{AMath 585}
\hbox{Take home final}\hbox{Due Tuesday, March 17, 2020}}

\vskip 5pt

Due to Canvas by 11:00pm PDT on the due date, at the latest.

No late papers accepted, so aim to get it in earlier!

To submit, see
\url{https://canvas.uw.edu/courses/1352870/assignments/5301318}

100 points possible.  

Open book and notes, and you can use other resources too, except please don't discuss it with other students or anyone else.

If you need clarification on some problem, or you think there's an error or
typo, please post it on the 
\href{https://canvas.uw.edu/courses/1352870/discussion_topics}{Canvas discussion board}
so everyone has access to the same information.

%--------------------------------------------------------------------------
\vskip 1cm
\hrule
{\bf Problem 1.}

In Homework 2 you solved a linear beam equation, valid for small deformations.
If the deformations are larger, then the equation must be nonlinear.  In some
cases an equation of this form can be used:
\begin{align*}
&a u''''(x) - b(u'(x))^2 (u''(x))^2 = f(x) \quad\text{for~} 0\leq x \leq 1\\
&u(0) = \alpha_0,~~u'(0) = \alpha_1, \quad
u(1) = \beta_0,~~u'(1) = \beta_1.
\end{align*}
where $a,~b,~\alpha_0,~\alpha_1,~\beta_0,~\beta_1$ are all specified constants.

(a) If we discretize with a uniform grid using $h=1/(m+1)$, suggest a 
nonlinear system of $m$ equations that could be solved for 
$[U_1,~U_2,~\ldots,~U_m]$ (the interior grid values) 
to obtain an approximate solution that is second order accurate.  

Make sure the boundary conditions are also handled appropriately.
Use the "First approach" described in 
\href{http://staff.washington.edu/rjl/classes/am585w2020/solutions/hw2\_solutions.html}{hw2\_solutions.html} 
since this is less messy and should still be second order accurate.


(b) If we wanted to solve this system using Newton's method, we would need
the Jacobian matrix for the nonlinear system developed in (a).  
You don't need to compute the full matrix,
but do compute the diagonal elements $J_{ii}$ for $i=1,~2,~\ldots,~m$.

% uncomment the next two lines if you want to insert solution...
%\vskip 1cm
%{\bf Solution:}

% insert your solution here!



%--------------------------------------------------------------------------
\vskip 1cm
\hrule
{\bf Problem 2.}

Suppose $A$ is a $2\times 2$ singular matrix that is symmetric and has one
postive eigenvalue, for example
\[
A = \bcm 1&2\\ 2&4\ecm
\]
is one such matrix.  Then $A$ is symmetric positive
{\em semi-definite} ($u^TAu \geq 0$ for all nonzero $u$) 
but is not positive definite.   If we define the functional
\[
\phi(u) = \frac 1 2 u^TAu - u^Tf
\]
as in Section 4.3 then levels sets of $\phi(u)$ are no longer ellipses.

(a) What is the geometry of these level sets in general?

(b) Let $z\in\reals^2$ be a null vector of $A$ and suppose $f\in\reals^2$ 
is a vector satisfying $z^Tf = 0$.  Then the system $Au=f$ has infinitely
many solutions.  Show that the steepest descent method applied to $\phi(u)$
from any initial guess $u^{[0]}$ converges to {\em some} solution of the linear
system in a single iteration.  Hint: It might be easier to explain this using 
the result of (a) than by computing $u^{[1]}$ explicitly.  That's fine as
long as you give a convincing argument.


% uncomment the next two lines if you want to insert solution...
%\vskip 1cm
%{\bf Solution:}

% insert your solution here!


%--------------------------------------------------------------------------
\vskip 1cm
\hrule
{\bf Problem 3.}

Consider the {\em first order ODE}, with a single boundary condition,
\begin{align*}
& u'(x) = f(x), \quad 0\leq x\leq 1,\\
&u(0) = \alpha.
\end{align*}
This boundary value problem has a unique solution 
$u(x) = \alpha + \int_0^x f(t)\, dt$. Using $u'(x_j) \approx
(U_j - U_{j-1})/h$ on a uniform grid, we might attempt to
approximate it by solving a system of the form
\[
\frac 1 h \brm 1\\-1&1\\ &-1&1\\ &&-1&1\\&&&\ddots&\ddots\\ &&&&-1&1\erm
\bcm U_0\\U_1\\U_2\\ U_3\\\vdots \\ U_m\ecm =
\bcm \alpha/h\\ f(x_1)\\ f(x_2)\\ f(x_3)\\ \vdots \\ f(x_m)\ecm.
\]

(a) Determine the exact solution to this linear system and show that 
$U_j$ approximates $\alpha + \int_0^{x_j} f(t)\, dt$ with a ``Riemann sum".

(b) Suppose we apply Gauss-Seidel, sweeping through the grid from left to right in the natural order (i.e. $j = 0, ~1,~\ldots,~m$).  
Explain why one iteration is sufficient to
converge to the exact solution of this linear system, for {\em any} choice
of initial data $U^{[0]}$.

(c) Suppose we instead sweep from right to left in
Gauss-Seidel (for $j = m,~m-1,~\ldots 0$).
What is the iteration matrix $G$ for this method on the system above?
What are the eigenvalues of the $G$ matrix? 
What are the matrices $G^2,~ G^3,~ \ldots$?  (There is a simple pattern.)

(d) Suppose we start with initial guess $U^{[0]}=0$ (the zero vector).
Does this backward Gauss-Seidel method converge in a single step? 
In a finite number of steps? If so, how many?

(e) In Section 4.2.1 we saw that the spectral radius $\rho(G)$ tells us 
something about the rate of convergence of the method.  Comment on 
what's going on in this example based on your answers to (c) and (d).

You are welcome to write a short code to try things out, but you are not
required to implement this.

% uncomment the next two lines if you want to insert solution...
%\vskip 1cm
%{\bf Solution:}

% insert your solution here!


%--------------------------------------------------------------------------
\vskip 1cm
\hrule
{\bf Problem 4.}

Consider the Conjugate gradient algorithm on page 87 for some symmetric positive
definite matrix $A \in\reals^{m\times m}$.  Suppose we happen
to choose the initial guess $u_0$ in such a way that the initial residual
$r_0 = f - Au_0$ is an eigenvector of $A$.  Show that in this case the
method converges to the true solution of the linear system in one iteration.

% uncomment the next two lines if you want to insert solution...
%\vskip 1cm
%{\bf Solution:}

% insert your solution here!


%--------------------------------------------------------------------------
\vskip 1cm
\hrule
{\bf Problem 5.} 

Consider the BVP 
\[
u''(x) = f(x), \qquad\text{for}~ 0\leq x \leq 1
\]
with boundary conditions
\[
\gamma_0 u(0) + \gamma_1 u'(0) = \sigma, \qquad u(1)=\beta.
\]
At $x=1$ a standard Dirichlet BC is specified, but at $x=0$ we now have a
``mixed'' or ``Robin'' boundary condition, assuming $\gamma_0,~\gamma_1,~\sigma$
are all specified constants, as is $\beta$. For some physical problems this is
the correct type of boundary condition, e.g. in a heat conduction problem it
corresponds to a case in which the heat flux at $x=0$ is related to the
temperature at this point.

\newpage
(a) Set up a tridiagonal linear system $Au=f$ that could be solved to
model this, with the following properties:
\begin{itemize}
\item $u = [u_0,~u_1,~\ldots,~u_m]$ contains the unknown boundary value $u_0$ 
but not the known value $u_{m+1} = \beta$ (assuming as usual that $x_j = jh$
for $j=0,~1,~\ldots,~m+1$ on a grid with $h=1/(m+1)$).
\item The method is second order accurate. 
\end{itemize}
Follow the strategy of the second approach on page 31 to obtain the first
equation in your linear system (i.e. introduce a ghost point $u_{-1}$ and 
then eliminate it from the two equations that involve this unknown).
Write out the matrix and right hand side of your system.

(b) Determine the local truncation error of your method, 
$\tau = [\tau_0,~\tau_1,~\ldots,~\tau_m]$.  We expect $\tau_j = C_jh^2 + o(h^2)$
so determine the constants $C_j$ in terms of derivative(s) of the exact solution
$u(x)$ (by doing Taylor series expansions, assuming $u(x)$ is sufficiently
smooth).

% uncomment the next two lines if you want to insert solution...
%\vskip 1cm
%{\bf Solution:}

% insert your solution here!



%--------------------------------------------------------------------------
\vskip 1cm
\hrule
{\bf Problem 6.} (With corrected boundary conditions)

(a) Implement the method you derived in the previous problem (in Python or Matlab).
It is ok to base this on code you have previously written for homework problems,
and/or the class Jupyter notebooks.

(b) Test it on the problem
\begin{align*}
&u''(x) = 4, \qquad 0\leq x \leq 1,\\
&2u(0) + 3u'(0) = 1, \qquad u(1)=2,
\end{align*}
which has exact solution $u(x) = 2x^2 + x - 1$.

Explain why you expect the exact solution to your linear system to agree with
exact solution of the ODE when evaluated at the grid points, and confirm that
this is true for your implementation.

(c) Also test it on the problem
\begin{align*}
&u''(x) = -18x + 4, \qquad 0\leq x \leq 1,\\
&2u(0) + 3u'(0) = 1, \qquad u(1)=-1,
\end{align*}
which has exact solution $u(x) = -3x^3 + 2x^2 + x - 1$.
In this case check that your method is second order accurate by producing
a log-log plot of the errors (similar to what was done in the notebook BVP1),
using $m+1 = 50,~100,~200,~400,~800$.


% uncomment the next two lines if you want to insert solution...
%\vskip 1cm
%{\bf Solution:}

% insert your solution here!


%--------------------------------------------------------------------------

\end{document}

