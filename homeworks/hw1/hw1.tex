\documentclass[10pt]{article}

\usepackage{graphicx}
\usepackage{amsmath,amsfonts,amssymb}

\usepackage{hyperref}  % for urls and hyperlinks


\setlength{\textwidth}{6.2in}
\setlength{\oddsidemargin}{0.3in}
\setlength{\evensidemargin}{0in}
\setlength{\textheight}{9.1in}
\setlength{\voffset}{-1in}
\setlength{\headsep}{26pt}
\setlength{\parindent}{0pt}
\setlength{\parskip}{5pt}




% a few handy macros

\newcommand\matlab{{\sc matlab}}
\newcommand{\goto}{\rightarrow}
\newcommand{\bigo}{{\mathcal O}}
\newcommand{\half}{\frac{1}{2}}
%\newcommand\implies{\quad\Longrightarrow\quad}
\newcommand\reals{{{\rm l} \kern -.15em {\rm R} }}
\newcommand\complex{{\raisebox{.043ex}{\rule{0.07em}{1.56ex}} \hskip -.35em {\rm C}}}


% macros for matrices/vectors:

% matrix environment for vectors or matrices where elements are centered
\newenvironment{mat}{\left[\begin{array}{ccccccccccccccc}}{\end{array}\right]}
\newcommand\bcm{\begin{mat}}
\newcommand\ecm{\end{mat}}

% matrix environment for vectors or matrices where elements are right justifvied
\newenvironment{rmat}{\left[\begin{array}{rrrrrrrrrrrrr}}{\end{array}\right]}
\newcommand\brm{\begin{rmat}}
\newcommand\erm{\end{rmat}}

% for left brace and a set of choices
\newenvironment{choices}{\left\{ \begin{array}{ll}}{\end{array}\right.}
\newcommand\when{&\text{if~}}
\newcommand\otherwise{&\text{otherwise}}
% sample usage:
%  \delta_{ij} = \begin{choices} 1 \when i=j, \\ 0 \otherwise \end{choices}


% for labeling and referencing equations:
\newcommand{\eql}{\begin{equation}\label}
\newcommand{\eqn}[1]{(\ref{#1})}
% can then do
%  \eql{eqnlabel}
%  ...
%  \end{equation}
% and refer to it as equation \eqn{eqnlabel}.  


% some useful macros for finite difference methods:
\newcommand\unp{U^{n+1}}
\newcommand\unm{U^{n-1}}

% for chemical reactions:
\newcommand{\react}[1]{\stackrel{K_{#1}}{\rightarrow}}
\newcommand{\reactb}[2]{\stackrel{K_{#1}}{~\stackrel{\rightleftharpoons}
   {\scriptstyle K_{#2}}}~}

% Parts:

% set enumerate to give parts a, b, c, ...  rather than numbers 1, 2, 3...
\renewcommand{\theenumi}{\alph{enumi}}
\renewcommand{\labelenumi}{(\theenumi)}

% set second level enumerate to give parts i, ii, iii, iv, etc.
\renewcommand{\theenumii}{\roman{enumii}}
\renewcommand{\labelenumii}{(\theenumii)}

  % input some useful macros

\begin{document}

% header:
\hfill\vbox{\hbox{AMath 585}
\hbox{Homework \#1}\hbox{Due Thursday, January 16, 2019}}

\vskip 5pt

Homework is due to Canvas by 11:00pm PDT on the due date.

To submit, see
\url{https://canvas.uw.edu/courses/1352870/assignments/5199959}


%--------------------------------------------------------------------------
\vskip 0.5cm
\hrule

If you haven't done so already, clone the class repository and read about how 
to use it on the class webpage
\url{http://staff.washington.edu/rjl/classes/am585w2020/class_repos.html}.

Also, this would be a good time to try to figure out how to install Python
and use Jupyter notebooks from the {\tt notebooks} directory
of the repository.  See 
\url{http://staff.washington.edu/rjl/classes/am585w2020/code.html}.
Contact the instructor or TA if you need help.



%--------------------------------------------------------------------------
\vskip 1cm
\hrule
{\bf Problem 1.}
Suppose we wish to approximate the fourth derivative $u^{(4)}(x_0)$ 
using a 5-point stencil
\[
u^{(4)}(x_0) \approx c_{-2} u(x_{-2}) + c_{-1} u(x_{-1}) + c_0 u(x_0) + c_1 u(x_1)
                   + c_2 u(x_2)
\]
in the case of equally spaced points, $x_j = x_0 + jh$ for some
$h = \Delta x$.

By hand, work out the Vandermonde system of equations to be solved for the
coefficients and confirm that the coefficients produced by the 
{\tt fdcoeffV} function gives coefficients that satisfy this linear system.
(Use either the Python or Matlab version, but I suggest you try using Python
and the Jupyter notebook.)


% uncomment the next two lines if you want to insert solution...
%\vskip 1cm
%{\bf Solution:}

% insert your solution here!

%--------------------------------------------------------------------------
\vskip 1cm
\hrule
{\bf Problem 2.}
(a) Determine the order of accuracy and leading term of the asymptotic error for
the approximation derived above.  In other words, if
the function $u(x)$ is sufficiently smooth, show the approximation has an error
that is of the form $C h^p + {\cal O}(h^{p+1})$ and determine $p$ and $C$.
The value of $C$ will depend on higher-order derivative(s) of $u$ at $x_0$.

(b) Test your result by computing the error for $u(x) = \sin(2x)$
at the point $x_0 = 1$ and various choices of $h$ and show this is
consistent with what you derived.  Produce a log-log plot of the absolute
error vs.\ $h$ and the expected error to show that it has the expected form 
(similar to the plots produced in the {\tt fdstencil\_errors.ipynb} notebook).
How good an approximation can you get before rounding error takes over?

% uncomment the next two lines if you want to insert solution...
%\vskip 1cm
%{\bf Solution:}

% insert your solution here!


%--------------------------------------------------------------------------
\vskip 1cm
\hrule
{\bf Problem 3.}
Suppose we want to solve a 2-point boundary value problem of the form
\[
u^{(4)}(x) = 3u''(x) + 4u(x) + f(x)
\]
for $a \leq x \leq b$ with prescribed boundary conditions 
\[
u(a) = \alpha_0, \quad u'(a) = \alpha_1, \qquad
u(b) = \beta_0, \quad u'(b) = \beta_1.
\]
Set up the linear system of equations that would be solved to find a
finite-difference solution to this problem.  For this homework you do not
need to program this or solve the system, just write out the system in
a way that is clear what the banded matrix is, and what system needs to be
solved. In particular, write
out explicitly at least the first three and last three rows of the matrix
and elements of the right-hand side to show how the function values $f(x_j)$
and boundary conditions come into these.

Use a uniform grid $x_j = a + jh$ with $h = (b-a)/(m+1)$, 
%including the boundary points
%$U_0 \approx u(a)$ and $U_{m+1} \approx u(b)$.
Use the approximation to $u^{(4)}$ from Problem 1 and the standard centered
approximation for $u''(x)$.  For the boundary conditions on $u'$, use 
one-sided approximations that are second-order accurate.

Since $u(a)$ and $u(b)$ are both known from the boundary conditions, you can
set this up as a system of $m$ equations for the interior unknowns $U_1,~
\ldots,~ U_m$.  


% uncomment the next two lines if you want to insert solution...
%\vskip 1cm
%{\bf Solution:}

% insert your solution here!


%--------------------------------------------------------------------------

\end{document}
